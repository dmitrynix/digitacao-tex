\documentclass[12pt,a4]{book}

\usepackage[brazil]{babel}
\usepackage[utf8]{inputenc}
\usepackage[T1]{fontenc}

\usepackage{graphicx}
\usepackage{fancyhdr}
\usepackage[colorlinks]{hyperref}

\lhead{}

\author{Dmitry Rocha}
\title{Apostila de Digitação}
\date{Projeto Iniciado em setembro de 2007 \\ Impresso em \today}

\begin{document}

\maketitle

\pagenumbering{arabic}
\pagestyle{fancy}
\tableofcontents

%\input{copyright}

\chapter{Digitação}

\section{Teclado}

Para a entrada de dados em computadores existem vários formatos de teclados:
\underline{Dvorak}, \underline{Qwert}, \underline{Qwert Abnt},
\underline{Qwert Abnt2}\footnote{semelhante ao \underline{Qwert Abnt} com a
adição da tecla \textsc{ç} ao lado da tecla \textsc{l}}.

Em todos estes o teclado é composto:

\begin{enumerate}
\item Das letras de caracteres de \textsc{a} até \textsc{z}, símbolos e
números (englobam as quatro fileiras principais do teclado);
\item De teclas para auxiliar a digitação:

\textsc{Tab}, para fazer tabulação, ou pular para o próximo campo/item;

\textsc{CapsLock}, que quando ativada todas as letras saem em maiúscula;

\textsc{Shift}, para colocar a letra, precionada em conjunto, para maiúscula,
colocar acentos, e que caso seja precionada juntamente com as setas, tem a
função de selecionar o texto, dispensando assim o uso do mouse;

\item Tecla \textsc{Control} determina atalhos do teclado, por exemplo:
formatação do texto;

\item Tecla \textsc{Alt} é mais usada para acessar menus e inserir um terceiro
caractere do teclado;

\item Teclas de funções: \textsc{F1} até \textsc{F12}, muito utilizadas
(sozinhas ou em conjunto com \textsc{Control}, \textsc{Shift} e \textsc{Alt}),
para agilizar o acesso a determinados recursos, exemplo: A ``Ajuda'' é
acessada, na maioria dos programas, com a tecla \textsc{F1};

\item Tecla \textsc{Insert} usada para ativar o modo para sobrescrever um
texto, i. e., posiciona-se o cursor no início do texto e vai-se digitando desta
forma o texto é substituido sem a necessidade de apagá-lo;

\item Tecla \textsc{Backspace} usada para apagar textos a esquerda do cursor, e
\textsc{Delete} para apagar texto a direita do cursor;

\item Tecla \textsc{Enter} para iniciar um parágrafo\footnote{Lembre-se:
muitos processadores/editores de texto iniciam uma nova linha assim que a atual
chega ao fim, desta forma somente é necessário entrar com um enter para
iniciar outro parágrafo.};

\item Teclas de movimentação: \textsc{setas}, \textsc{Home},
\textsc{Page Down/Up}, \textsc{End}, \ldots

\end{enumerate}

\section{Posicionamento dos Dedos}

\subsection{Parte Alfanumérica do Teclado}

O correto posicionamento dos dedos e das mãos é o fator decisivo para
conseguir fluidez na digitação.

\begin{enumerate}

\item Em todos os teclados existem marcas salientes que funcionam como ganchos
para os dedos não sairem do lugar e permanecerem sempre na mesma posição;

Para o teclado \underline{Qwert Abnt} e \underline{Qwert Abnt2}\footnote{Os
mais comuns no Brasil, e, este último, o que provavelmente você está usando.}:
os dedos indicadores permanecem repousados sobre a tecla \textsc{F}, para mão
esquerda, e sobre o \textsc{J}, mão direita;

No teclado \underline{Dvorak} elas repousam sobre \textsc{U} e \textsc{H};

\item Os dedos polegares repousam sobre a barra de espaço;

\item Os outros dedos seguem em sequência e na mesma linha dos dedos
indicadores;

\item Para ter um melhor conforto com a digitação a base do pulso
permanece sobre a mesa, com isso não cansa os braços.

\end{enumerate}

Conforme uma tecla seja necessária o dedo responsável por ela, e
somente ele, deve mover-se, retornando para sua posição inicial logo
que terminado.

\textit{Depois de se incorporar profundamente essa relação na memória, pode-se
relaxar mais a obediência à regra anterior, de modo a possibilitar uma maior
velocidade para o pressionamento das teclas} -
Klavaro, Software de Digitação.

\subsection{Parte Numérica do Teclado}

Da mesma forma para digitar letras o teclado numérico tem um gancho no número 5.

\section{Ergonomia}

O uso excessivo do computador pode causar problemas a saúde, entre os problemas
o mais comum é: \emph{LER}, Lesões por Esforço Repetitivo.

Muitos utilizam o computador para acesso rápido, mas uma cadeira
desconfortante, um monitor inadequado ou a ausência de apoio para as mão causam
desconfortos e problemas físicos.

\subsection{O Pescoço, Os Olhos e a Região Lombar}

Em média as dimensões da cabeça humana (adulto) são as mesmas da de uma bola de
boliche (5,5 kg). O pescoço suporta seu peso facilmente, mas quando ela é posta
para frente ou para trás, os músculos esticam-se ou contraem para suporta-lá,
causando dores incomodas.

Uma das maiores reclamações dos usuários é a fadiga nos olhos (pressão na
vista, olhos ressecados, lacrimação e visão cansada). Muitas vezes simples
alterações no ambiente de trabalho pode eliminar alguns ou todos os problemas.

O olho humano limpa-se e refresca-se por si mesmo, automaticamente, várias vez
por dia. Entretanto, o ato de concentrar a atenção durante muito tempo no
brilho do monitor causa uma diminuição significava no piscar de olhos. Fazer
paradas frequentes enquanto se utiliza o computador e piscar os olhos ajudam a
relaxar e refrescar a vista.

Outro fator que contribui para a fadiga dos olhos resulta da exaustão muscular.
Pequenos músculos são responsáveis por mudanças no formato de suas lentes, para
permitir a aproximação e o distanciamento da visão. A menos que estejam
relaxados, esses músculos produzem ácido láctico, que provoca fadiga. Olhar
através da janela ou da sala faz com que os músculos se afrouxem e os olhos
recebam sangue oxigenado, removendo assim o ácido láctico.

Cadeira inadequada é o maior causador de dores nas costas. O encosto da cadeira
precisa estar posicionado exatamente na curvatura lombar, fazendo com que a
coluna se mantenha apoiada. O encosto também deve ser flexível a ponto de não
permitir que o usuário escorregue para trás.

\section{Programas que Trabalham com Texto}

\subsection{Editores de Texto}

Programas que permitem editar texto sem que seja feito qualquer formatação.
Somente diferenciando linhas e parágrafos.

Exemplos: Gedit, Gvim, Bloco de Notas, \ldots

\subsection{Processadores de Texto}

Programas que permitem além de inserir texto a formatação desse texto (tamanho,
fonte, \ldots). Permitem inserir tabelas (semelhante as de uma planilha
eletrônica).

Exemplos: \mbox{OpenOffice.org Writer}, \mbox{Microsoft Word}, \ldots

\subsection{Planilhas Eletrônicas}

Tabelas são a forma para a qual pode ser armazenada cada valor em uma célula
tornando fácil sua manipulação. Pode ser inserido qualquer tipo de valores
letras, números, símbolos\ldots Além de poder fazer facilmente cálculos com
elas: médias, somas, multiplicação, resultado condicional\ldots

Exemplos: \mbox{OpenOffice Calc}, \mbox{Microsoft Excel}, \ldots

\subsection{Softwares de Apresentação}

Programas como \mbox{OpenOffice.org Impress} e \mbox{Microsoft PowerPoint}
podem fazer apresentações que pode servir para expor idéias em forma slides.

\end{document}
